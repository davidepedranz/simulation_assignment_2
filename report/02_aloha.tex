\section{Aloha}
\label{sec:aloha}

A \ac{MAC} protocol defines how stations access a shared channel in order to transmit messages to the others.
Aloha is a very simple \ac{MAC} protocol developed by Norman Abramson and colleagues at the University of Hawaii in the 1970s to realize a broadcasting communication between nodes spread on different islands in the archipelago. 

The behaviour of the original Aloha is the following: when a packet arrives to the station, it is immediately transmitted to all stations in the communication range.
When idle, the station listens for variations of energy in the channel in order to be able to detect and receive incoming packets.
After transmitting or receiving a packet, a small amount of time is required to process it.
If new packets arrive while the station is already transmitting, receiving or processing a packet, they are queued and will be transmitted as soon as the station finishes the current action.
If multiple packets are transmitted at the same time, they are very likely to collide and be corrupted.

Aloha does not try to avoid or detect collisions.
In case of collisions, the packet are not recognizable and thus simple discarded.
In other words, there is no guarantee that a packet is correctly received by the other stations.
