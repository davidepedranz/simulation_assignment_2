\section{Realistic Propagation}
\label{sec:realistic_propagation}

The first implemented change is the packets' propagation model.
The original simulator uses a disk model: a packet is always correctly received from all and only the stations in the range of the transmitter if no collision occur.
More formally, the receiving probability for a packet is modelled as follows:

\begin{equation}
Pr(correct\ |\ d) =
    \left\{
    	\begin{array}{ll}
    		1 & \mbox{if } d < RX_{range} \\
    		0 & \mbox{otherwise}
    	\end{array}
    \right.
    \label{eq:original}
\end{equation}

In the real world, however, signals soften with the distance.
A simple model of this phenomena is the following:

\begin{equation}
Pr(correct\ |\ d) =
    \left\{
    	\begin{array}{ll}
    		1 - \frac{d}{RX_{range}} & \mbox{if } d < RX_{range} \\
    		0 & \mbox{otherwise}
    	\end{array}
    \right.
    \label{eq:realistic}
\end{equation}

It is possible to specify the model to use in the simulator using the \texttt{propagation} parameter in the configuration file.
Allowed values are \texttt{original} for the model of \cref{eq:original} or \texttt{realistic} for the model of \cref{eq:realistic}.

If the realistic model is selected, the channel tags each packet with the probability of correct receiving for the receiver.
When a node processes a not corrupted packet, a random variable from a Uniform distribution between $0$ and $1$ is extracted: if the value is higher or equal to the probability of correct receiving, the packet is marked as correctly received, otherwise it is marked as corrupted.
The simulator distinguish between ``corrupted for a collision'' and ``corrupted because of the distance'' to allow a detailed analysis of the results.
